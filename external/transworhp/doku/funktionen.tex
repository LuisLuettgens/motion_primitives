

\chapter{Funktions�bersicht}

In diesem Kapitel werden die wichtigsten Funktionen erkl�rt.

\section{Die Klasse {\tt TransWorhp}}

Von dieser Klasse werden alle Probleme abgeleitet.

\subsection{Systemmethoden}

Konstruktor 

\syntax{TransWorhp(const std::string \&s, int dis, int ode, int ctrl, int param, int rand, int neben);}

Destruktor

\syntax{\textasciitilde TransWorhp();}

WORHP-Arbeitsspeicher anlegen und mit Instanz verkn�pfen:

\syntax{int Init(XMLNode *xml, OptVar \&o, Workspace \&w, Params \&p, Control \&c, Viewer *viewer=0);}

XML-Dateien lesen:

\syntax{static XMLNode *ReadParams(const std::string\& filename);}

Optimierungsschleife:

\syntax{int Loop();}


\subsection{Zugriffsmethoden}

Die Zust�nde und Steuerungen werden in den Optimierungsvektor von WORHP geschrieben. Damit komfortabel auf einzelne von ihnen zugegriffen werden kann, stehen verschiedene Funktionen bereit. Steuerung und der Zustand lassen sich zu jedem diskreten Zeitpunkt {\tt dis} mit folgenden Methoden auslesen. Da freie Parameter f�r alle Zeitpunkte gleich sind, gibt es nur einen Wert.

\syntax{double x(int dis, int ode) const;

	double u(int dis, int ctrl) const;

	double p(int param) const;
}

Wird das Optimalsteuerungsproblem mit Verfahren h�herer Ordnung (z.B Hermite-Simpson) gel�st, werden Zwischenpunkte eingef�gt. An diesen Punkten l�sst sich ebenfalls der Zustand und die Steuerung auslesen. Die Methoden {\tt x\_\_()} und {\tt u\_\_()} laufen �ber alle Punkte, wohingegen die Methoden {\tt x()} und {\tt u()} nur an den 'Haupt'punkten auswerten und alle Zwischenpunkte �berspringen.

\syntax{double x\_\_(int dis, int ode) const;

double u\_\_(int dis, int ctrl) const;
}

Indexbestimmung 

\syntax{
	int x\_index(int dis, int ode);

	int u\_index(int dis, int ctrl);

	int p\_index(int param);

	int p\_indexode(int param);
}
	
\subsection{Implementierung des OCP}

\subsubsection{�berladbare Methoden}


Zielfunktion 

	\syntax{double obj();

bool obj\_structure(DiffStructure \&s);

bool obj\_diff(DiffStructure \&s);
}
	
	
ODE-System

	\syntax{
void ode(double *dx, double t, const double *x, const double *u, const double *p);

	bool ode\_structure(DiffStructure \&s);

	bool ode\_diff(DiffStructure \&s, double t, const double *x, const double *u, const double *p);

	bool ode\_diff\_p(DiffStructure \&s, double t, const double *x, const double *u, const double *p, int index);
	}

(letztere Funktion bestimmt ausschlie�lich die Ableitung nach Parametern!)


Box-Beschr�nkungen

	\syntax{void x\_boundary(double *x\_low, double *x\_upp);

	 void u\_boundary(double *u\_low, double *u\_upp);

	 void p\_boundary(double *p\_low, double *p\_upp);

	 void var\_boundary(double *x\_low, double *x\_upp);
}

Rand-Bedingungen

\syntax{	 void rand(double *r);

	 void rand\_boundary(double *r\_low, double *r\_upp);

	 bool rand\_structure(DiffStructure \&s);

	 bool rand\_diff(DiffStructure \&s);
}

	Nebenbedingungen

	\syntax{void neben(double *c, double t, const double *x, const double *u, const double *p);

	 void neben\_boundary(double *c\_low, double *c\_upp);

	 bool neben\_structure(DiffStructure \&s);

	 bool neben\_diff(DiffStructure \&s, double t, const double *x, const double *u, const double *p);

	 bool neben\_diff\_p(DiffStructure \&s, double t, const double *x, const double *u, const double *p, int index);
	}

	Startsch�tzung

\syntax{ void init();

	 void p\_init(double *p);

	 void x\_init(double *x, int i, int dis);

	 void u\_init(double *u, int i, int dis);
}

\subsection{I/O-Methoden}

\subsubsection{�berladbare Methoden}

Aufruf in jedem Iterationschritt f�r Konsolen- oder Datei-Ausgabe:

\syntax{	
int step();
}
Wird 0 zur�ckgegeben, wird die Optimierung unterbrochen.

Aufruf am Ende der Optimierung f�r Konsolen- oder Datei-Ausgabe:

\syntax{	
void terminate();
}


\subsubsection{Grafisches Interface}

Weitere Plots erzeugen und Plots beschriften:

\syntax{void OpenWindows(Viewer *gr);

std::string GetXTitle(int d);

std::string GetUTitle(int d);
}

\subsection{Weiterf�hrende Methoden}

Zeitdiskretisierung anpassen, z.B. im Konstruktor:	

\syntax{void TimeAxis(double exponent);}
oder manuell.





Zustand oder Steuerung an beliebigem Zeitpunkt auslesen:

\syntax{
	void GetState(double *x, double t);

	void GetControl(double *u, double t);
	}


\section{Die Struktur {\tt DiffStructure}}

Eintr�ge in den Ableitungsmatrizen k�nnen angelegt und gesetzt werden durch den Klammeroperator:

\syntax{
	double\& operator()(int i, int j);
}

\subsection{Zugriff auf DiffStructure}

Mit Hilfsfunktionen wird der richtige Index zugeordnet.

lokale Strukturen (ode, integral, neben):
Zugriff mit x\_indexode

globale Strukturen (rand, obj):
Zugriff mit x\_index

TODO: Bild

\section{Die Klasse {\tt Viewer}}

In {\tt OpenWindows()} k�nnen neue Plots hinzugef�gt werden:

Struktur der Ableitungs-Matrix plotten

\syntax{
	void Matrix(const std::string \&s, WorhpMatrix *m);
}

Daten plotten, die in einer Funktion {\tt func} bereitgestellt werden:

\syntax{
	void Data(const std::string \&s, Funktionenzeiger2 func, int lo, int hi, int index);
}

Dabei ist {\tt func} von diesem Typ:
 
\syntax{
typedef double (*Funktionenzeiger2) (int \&len, int \&ndgl, int \&nsteuer,
                                     double *t,double *x, double *u, int \&i, int \&index);
}
{\tt index} kann zur internen Unterscheidung verwendet werden.

Phasen-Diagramm plotten (also Daten {\tt func(..., d1)} gegen {\tt func(..., d2)}).

\syntax{
	void PhasePlot(const std::string \&s, Funktionenzeiger2 func, int d1, int d2);
}

OpenGL-Plot hinzuf�gen.

\syntax{
	void ThreeD(const std::string \&s, XMLNode *xml, plot3d f);
}

Dabei ist {\tt f} von diesem Typ:

\syntax{
typedef void (*plot3d) (glObject *obj, double *x, double t);
}

\example{Animation von Objekten

Modell mit Cinema4D erstellen und mit riptide exportieren. 
}

Vergleichswerte hinzuf�gen:

\syntax{
	void AddCompareCurve(int \&cmpstep, double* cmp, double* cmptime, int n);
}

Alle Plots schlie�en oder neu anordnen:

\syntax{
	void CloseAll();
	void TilePlots();
}


%	void SetHeader(char *s, int mode);

%	void AutoScale();


%	size_t Size() const;

%	void SetFloatTime(double t);
%	void SetTimeIsTime();

		
\section{Die Klasse {\tt MagicDouble}}
\label{magicdbl}

Automatische Berechnung der 1. und 2. Ableitung.

TODO
